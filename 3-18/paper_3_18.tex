\documentclass[]{IEEEtran}
% some very useful LaTeX packages include:
%\usepackage{cite}      
\usepackage{graphicx}   
\usepackage{subfigure} 
\usepackage{url}       
\usepackage{amsmath}    
\usepackage{caption2}
% Your document starts here!
\begin{document}

% Define document title and author
	\title{Weekly Report}
	\author{Adviser: Prof. Yang Wen \\Student: Cheng Wensheng\\ Period: 2018.3.12- 3.18
	}
	\markboth{Visual Information Processing Group}{}
	\maketitle

% Write abstract here
\begin{abstract}
	This week I mainly put my effort on training DeepLab V3 and FC-Densenet model on our dataset.
\end{abstract}

% Each section begins with a \section{title} command
\section{DeepLab V3}
	% \PARstart{}{} creates a tall first letter for this first paragraph
	\PARstart{D}{eepLab} V3 is a state-of-the-art semantic segmentation model, released by Google. Since DeepLab V2 hasn't achieved a good performance on our dataset, so I turn to V3.
	\begin{itemize}
		\item Recently, Google has released official DeepLab series code on GitHub. But it only supports 2 public datasets, PASCAL VOC and CityScapes till now, and doesn't support customer's dataset.
		\item After searching a lot on GitHub, I found one open source project called \textbf{Semantic Segmentation Suit}e. It implements most fashion models, including \textbf{DeepLab V1-V3+}, \textbf{PSPNet}, \textbf{RefineNet}, \textbf{FC-DenseNet}, etc.
		\item After training DeepLab V3 model for 100 epochs, I get the result on training set as follows. The ground truth image is Fig.~\ref{fig:dlv3_gt}, the prediction image is Fig.~\ref{fig:dlv3_pr}.
	\end{itemize}

% Main Part
\section{FC-DenseNet}
	% LaTeX takes complete care of your document layout ...
	DenseNet wins the CVPR 2017 best paper. It improves ResNet significantly. FC-DenseNet combines fully connected net with DenseNet, and get state-of-the-art result. I tried this as well.
	\begin{itemize}
		\item The training time of FC-DenseNet is about 2 times as DeepLab V3 based on ResNet, maybe because the former structure is more complicated than the latter.
		\item After training FC-DenseNet for 100 epochs, I get the following result. The ground truth image is Fig.~\ref{fig:fcd_gt}, the prediction image is Fig.~\ref{fig:fcd_pr}. Better than DeepLab V3.
		\item Since the author has updated some model files, I'm training them with extended data again to see whether it would be better.
	\end{itemize}
\newpage
\begin{figure}[!hbt]
%		 Center the figure.
		\vspace{2.5cm}
%		\hspace{50cm}
		\begin{center}
			\begin{minipage}[t]{0.21\textwidth}
				\includegraphics[width=\columnwidth]{dl3_SAT01-00008_gt}
				%		 Create a subtitle for the figure.
				\caption{Ground truth image DeepLab V3}
				\label{fig:dlv3_gt}
			\end{minipage}
		    \hspace{0.5cm}
			\begin{minipage}[t]{0.21\textwidth}
				\includegraphics[width=\columnwidth]{dl3_SAT01-00008_pred}
				%Create a subtitle for the figure.
				\caption{Prediction image DeepLab V3}
				\label{fig:dlv3_pr}
			\end{minipage}
		\end{center}
	\end{figure}

\begin{figure}[!hbt]
	%		 Center the figure.
%	\vspace{1.8cm}
	\begin{center}
		\vspace{-0.3cm}
		\begin{minipage}[t]{0.21\textwidth}
			\includegraphics[width=\columnwidth]{fc_SAT01-00008_gt}
			%		 Create a subtitle for the figure.
			\caption{Ground truth image FCDenseNet}
			\label{fig:fcd_gt}
		\end{minipage}
		\hspace{0.5cm}
		\begin{minipage}[t]{0.21\textwidth}
			\includegraphics[width=\columnwidth]{fc_SAT01-00008_pred}
			%Create a subtitle for the figure.
			\caption{Prediction image FCDenseNet}
			\label{fig:fcd_pr}
		\end{minipage}
	\end{center}
\end{figure}

% Now we need a bibliography:
%\begin{thebibliography}{5}
%
%	%Each item starts with a \bibitem{reference} command and the details thereafter.
%	\bibitem{HOP96} % Transaction paper
%	J.~Hagenauer, E.~Offer, and L.~Papke. Iterative decoding of binary block
%	and convolutional codes. {\em IEEE Trans. Inform. Theory},
%	vol.~42, no.~2, pp.~429–-445, Mar. 1996.
%
%	\bibitem{MJH06} % Conference paper
%	T.~Mayer, H.~Jenkac, and J.~Hagenauer. Turbo base-station cooperation for intercell interference cancellation. {\em IEEE Int. Conf. Commun. (ICC)}, Istanbul, Turkey, pp.~356--361, June 2006.
%
%	\bibitem{Proakis} % Book
%	J.~G.~Proakis. {\em Digital Communications}. McGraw-Hill Book Co.,
%	New York, USA, 3rd edition, 1995.
%
%	\bibitem{talk} % Web document
%	F.~R.~Kschischang. Giving a talk: Guidelines for the Preparation and Presentation of Technical Seminars.
%	\url{http://www.comm.toronto.edu/frank/guide/guide.pdf}.
%
%	\bibitem{5}
%	IEEE Transactions \LaTeX and Microsoft Word Style Files.
%	\url{http://www.ieee.org/web/publications/authors/transjnl/index.html}
%
%\end{thebibliography}

% Your document ends here!
\end{document}